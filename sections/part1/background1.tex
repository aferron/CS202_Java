\documentclass[../../main.tex]{subfiles}

\begin{document}
\subsection{What is Java?}
The most simple answer to this question would be: Java is a platform
independant programming language.
Although, this actually doesn't describe Java well.
\\\\
Java is more comletely described as the Java Development Kit (JDK),
a software platform made up of various components such as the JVM
and JRE.
This seems like a lot at first, but it actually isn't too bad. In order
to start developing in Java one has to download the JDK which includes
tools like the Java compiler/debugger and the Java Runtime Environment
(JRE). The JRE is a piece of software that contains everything needed
to run a compiled Java program (called a Java Bytecode - similar to a
C++ binary) including the Java libraries, the Java Virtual Machine
(JVM), and deployment tools such as browser plugins so your Java
program can be run over networks and inside browsers without you having
to worry about gritty low level implementation details. The JVM is
basically a simulated CPU that can be installed on a multitude of
platforms. This is how Java achieves platform agnosticism.
\\\\
Another benefit of the Java runtime is that it takes care of memory
management through a process called garbage collection (GC). This
provides the develper the oppurtunity to focus on other aspects of
program design and can prevent issues like memory leaks and memory
based security exploits. This allows for the development complex
systems with improved reliability and safety.
\\\\
There are some drawbacks of large runtimes like the JRE. The most
noteworthy being inconsistent performance.
You don't know when the GC will run and if you are developing
performance critical code, think airplanes or stoplights, then Java
might not be the best tool for you. Java uses an incremental GC
system so these inconsistencies are almost always negligible. Another
issue is long start up time. The runtime is so large it takes a minute 
to get going. This issue has been quite the challenge for developers at
Oracle to address. When you start using IntelliJ you will understand
exactly what I mean and avoid closing the IDE at all costs. Lastly, if 
there is a bug in the runtime there is very little you can do to fix
your software. Although, given the incredible complexity of modern
compilers, similar issue exist within languages that don't provide
runtimes.

\subsection{Why use an IDE now?}
IDE's contain tools that are incredibly useful for developing large projects by providing tools like program frameworks,
continuous integration systems, and static or dynamic analysis. You won't be using any of these tools for some time but
it is important to start getting used to different development environments that you may have to use throughout your
education and careers. It is important to have a strong base of knowledge before utilizing these tools as to not become
too dependent on them. Check out the nearly 4000 plugins available for IntelliJ from their plugin repository and you
will likely find something that will contribute to your workflow. One plugin that is particulary useful is the Vim plugin,
which updates the key bindings and provides the same multi-mode interface. It even allows you to source a vimrc for further
customization.

\subsection{Vocabulary}
Java developers use some different vocabulary that is worth being
aware of:
   \begin{itemize}
      \item Member: A Method or a Field of a class.
      \item Methods: Are functions of a class. If you use the words
         "member function" to a Java developer they might laugh at you.
      \item Fields: A piece of data in a class (like an int or a
         String). These are sometimes called variables as well but a
         variable also refers to local data where a field refers to a
         class member.
      \item Referance: Its a really fancy pointer. It is very differen
         from a C++ referance and much closer to a C++ pointer.
         (a C++ referance is immutable and CANNOT be null)
      \item final: A Java keyword that designates something cannot be
         changed. A final class cannot be derived from, a final
         variable is like a C++ const variable, and a final method
         cannot be overridden.
      \item static: A Java keyword that is the same as C++ but is worth
         noting. A static field is shared by all instances of the class
         and a static method is called using the class itself, not an
         instance of the class. Static methods can be invoked even when
         no objects of that class have been instantiated and can only
         operate on static fields. Static methods and variables are
         also often called class methods and class variables.
      \item super: A Java keyword to access members of a class's
         parent.
      \item Supertype (of type A): Any class or interface above A in
         the inheritence tree.
      \item this: A Java keyword to represent the instance of the class
         in which it appears. Although Java does not require, it is
         idiomatic in Java to use this.member whenever you are
         accessing members within a class.
      \item extends: A Java keyword to declare that a class is a
         subclass of another.
      \item interface: A Java keyword used to define a collection of
         methods and constant values. Simlar to an abstract base class
         with some subtle differences. Interfaces can be used as types
         and are heavily utilized in the Java Collections.
   \end{itemize}

\subsection{Tips for Success}
Use Oracle's documentation. The Java version at time of writing is Java
SE 13 with the most recent LTS version being Java 8.
You probably downloaded the the SE 13 version of the JDK. The API
documentation for SE 13 is found here:
\vspace{2em}
https://docs.oracle.com/en/java/javase/13/docs/api/index.html
\vspace{2em}
Oracle's documentation is incredibly detailed and organized. Any question you have about the language can be answered by
reading them. In the process of looking for the answer you will learn things about the language and develop skills that are not
limited to Java. The tutorials page in the documentation is also full of easy to read and detailed explanations of various topics.
Stack Overflow and other tutorial websites can be useful to gather information but be careful to assess the quality. Try using
these resources to find where in the documentation to look instead of massaging other's code examples.
\\\\
In extension to the previous piece of advice, consider using the Java collections. Program 5 details that you have to construct
some of your own data structures from scratch but you can use the libraries for anything else you would like. I suggest looking
at the ArrayList and HashMap classes.

\subsection{Using IntelliJ Idea}
Copy in this subsection from old lab book, these steps seem mostly the same and is useful information.
\end{document}
