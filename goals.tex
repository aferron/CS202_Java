\documentclass[../../main.tex]{subfiles}

\begin{document}

\begin{steps}
   \item Research what IDE to install and complete the prelab by installing and configuring an IDE
   \item Experience using an IDE and tools it provides like the debugger, compilation, and Git
      integration
   \item Become familiar with the general Java vocabulary and idiomatic practices
   \item Start the lab with 10 minutes from the Linux and Vim manual
   \item Experience implementing Java software starting with basic concepts like:
   \begin{enumerate}[label=\Alph*.]
         \item I/O
         \item Primitive types
         \item Functionality of main
         \item Instantiating class objects
         \item Inheritance
         \item Recursion
      \end{enumerate}
   \item Once everything is completed, move on to the optional advanced topics of:
   \begin{enumerate}[label=\Alph*.]
         \item Using Java collections and iterators and creating your own generic types
         \item Usage of static and initialization blocks
         \item Interfaces versus Abstract Classes and psuedo-multiple inheritance
      \end{enumerate}
\end{steps}

\vspace{5cm}

\begin{center}
   {\Huge\bfseries IMPORTANT}
\end{center}

\fbox{
   \parbox{\textwidth}{
      \textbf{
         If possible bring your own laptop to these labs so you can learn to
         install and work with an IDE. A couple of popular IDEs for Java
         development are IntelliJ and Eclipse. Use IntelliJ if you want IDE
         help from the TCSS. Some of IntelliJ's qualities include its robust
         ecosystem, intuitive interface, and reliability.
      }
   }
}

\end{document}
